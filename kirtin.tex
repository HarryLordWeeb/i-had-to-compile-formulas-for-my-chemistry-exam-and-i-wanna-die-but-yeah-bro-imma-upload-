\documentclass[11pt]{Article}
\usepackage{amsmath}
\usepackage{amsfonts}
\usepackage{amssymb}
\usepackage[russian]{babel}

\begin{document}




\section{Електроотрицателност}
\begin{center}
{\Huge $\chi = \frac{I_A+F_A}{2}$ }\\
\end{center}
\newpage

\section{Куонова стабилизация}

\begin{center}
{\Huge$U=\frac{N \cdot e^2 \cdot z_1 \cdot z_2}{r} $}\\
\end{center}
\newpage

\section{Междумолекулно взаимодействие}
\begin{center}
{\Huge$ RT = (V-b)\cdot(p + \frac{a}{V^2}) $}\\
\end{center}
\newpage

\section{Ксемови сили}

\begin{center}
{\Huge$E_k = -\frac{2\mu^2_1 \cdot \mu^2_2}{3\cdot K \cdot T \cdot r^6} $}\\
\end{center}
\newpage

\section{Индукционни(Дебаеви) сили}

\begin{center}
{\Huge$E_D = - \frac{2\alpha \cdot \mu^2}{r^6} $}\\
\end{center}
\newpage

\section{Дисперсионни сили(лондонови)}
\begin{center}
{\Huge $E_L =-\frac{3\alpha \cdot h \cdot\ \nu_0}{4\cdot r^6} $ }\\
\end{center}
\newpage

\section{Междомолекулно привличане}
\begin{center}
{\Huge $E^I_V =E_K + E_D + E_L = \frac{- \frac{2\mu^2_1 \cdot \mu^2_2}{3 \cdot K \cdot T}- 2 \cdot \alpha \cdot \mu^2 - \frac{3 \cdot \alpha^2 \cdot h \cdot \nu_0 }{4}} {r^6} = \frac{n}{r^6}$ 
}\\

\end{center}
\newpage

\section{Силно сближени Particles}
\begin{center}
{\Huge $E^{II}_V = \frac{m}{r^{12}} $}\\
\end{center}
\newpage

\section{Пълни Вандервалсови сили}
\begin{center}
{\Huge $E_V = E^I_V + E^{II}_V = -\frac{n}{r^6} + \frac{m}{r^{12}} $}\\
\end{center}
\newpage

\section{Дисосиционна константа (пример с молекула HA)}
\begin{center}
{\Huge $ HA\leftrightarrows H^{+} + A^{-}  \Rightarrow K_\alpha = \frac{[H^{+}]+[A^{-}]}{[HA]} $}\\

\end{center}
( В квадратни скоби се означава моларна концентрация. Ако има два мола от нещо, то се степенува)
{\Huge $ M + nL \leftrightarrows MLn \Rightarrow \beta = \frac{[MLn]}{[M]\cdot[L]^n} $ }\\
\newpage

\section{pH}
\begin{center}
{\Huge $pH = -log(C_{H+}) $} \\ 
\end{center}
С- Концентрация 
\newpage

\section{Масово съдържание}
\begin{center}
{\Huge $\omega_{(x)} = \frac{m_{(x)}}{m}\cdot 100 = $}\\
\end{center}
\newpage

\section{Обемно съдържание}
\begin{center}
{\Huge $\omega_{(x)} = \frac{m_{(x)}}{V \cdot p} \cdot 100 $}\\
\end{center}
\newpage

\section{Молно съдържание}
\begin{center}
{\Huge $N_1 = \frac{n_1}{n_1 + n_2} and N_2 = \frac{n_2}{n_1 + n_2} $}\\
\end{center}
\newpage

\section{Масова концентрация}
\begin{center}
{\Huge $\rho_{(x)} = \frac{m{(x)}}{V} $}\\ g/l
\end{center}
\newpage

\section{Моларна концентрация}
\begin{center}
{\Huge $C_{M(x)} = \frac{n_{(x)}}{V}$}\\ mol/l
\end{center}
\newpage

\section{Молялна концентрация}
\begin{center}
{\Huge $C_{M(x)} = \frac{n_{(x)}}{m_0}$}\\ mol/kg
\end{center}
\newpage

\section{Нормална концентрация}
\begin{center}
{\Huge $C_H =\frac{n_{ekv}}{V}$}\\
\end{center}
\newpage

\section{Флотация}
{\Huge $cos\theta = \frac{\sigma_{2,3} - \sigma_{1,3} }{\sigma_{1,2}}$}\\ 1- течност 2- въздух- 3 твърдо в-во


to be continued...

\end{document}
